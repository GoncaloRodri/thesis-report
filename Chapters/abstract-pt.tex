%!TEX root = ../template.tex
%%%%%%%%%%%%%%%%%%%%%%%%%%%%%%%%%%%%%%%%%%%%%%%%%%%%%%%%%%%%%%%%%%%%
%% abstract-pt.tex
%% NOVA thesis document file
%%
%% Abstract in Portuguese
%%%%%%%%%%%%%%%%%%%%%%%%%%%%%%%%%%%%%%%%%%%%%%%%%%%%%%%%%%%%%%%%%%%%

\typeout{NT FILE abstract-pt.tex}%


A Internet tornou-se uma ferramenta indispensável para aceder à informação e exercer a liberdade de expressão, um direito fundamental que confere poder aos indivíduos em todo o mundo. No entanto, os regimes autoritários e as entidades governamentais tendem a restringir este direito através de práticas de vigilância e censura. Estes esforços têm-se tornado cada vez mais sofisticados, recorrendo a técnicas como a análise de tráfego e métodos avançados de correlação de dados para monitorizar e comprometer o anonimato dos utilizadores.

Em resposta à crescente necessidade de comunicações seguras e privadas, surgiram redes de anonimato como o Tor, que desempenham um papel essencial na proteção das comunicações online e na resistência à censura, das quais milhões de pessoas dependem diariamente. No entanto, investigações recentes demonstraram que ataques de desanonimização podem ser conduzidos por adversários poderosos a nível estatal, utilizando correlação de tráfego e técnicas avançadas de aprendizagem automática para identificar padrões e comprometer a privacidade dos utilizadores.

Esta dissertação aborda estes desafios, reforçando as defesas da rede Tor contra os ataques supramencionados. O principal objetivo é desenvolver uma solução inovadora para melhorar os nós de retransmissão Tor (Relay Nodes), aproveitando a arquitetura de software existente e introduzindo garantias formais de privacidade baseadas em circuitos privados diferenciados. Pretende-se que a solução proposta preserve níveis práticos de throughput e latência para comunicações de ponta a ponta, garantindo a compatibilidade com cenários de utilização real e proporcionando aos utilizadores uma proteção do anonimato mais robusta e adaptável.

Como referido anteriormente, o cerne da solução proposta assenta na integração dos princípios da Privacidade Diferencial, introduzindo dinamicamente ruído aleatório cuidadosamente delimitado nos fluxos de tráfego, mitigando os ataques de correlação e reforçando a resistência à recolha de impressões digitais. Tanto quanto é do nosso conhecimento, os contributos esperados desta dissertação poderão representar um avanço pioneiro na incorporação da Privacidade Diferencial na arquitetura de software dos nós de retransmissão Tor (Relay Nodes).
Além disso, a aplicação da Privacidade Diferencial permitirá o desenvolvimento de um mecanismo formalmente comprovado, escalável e eficaz, que reforça o anonimato online e se mantém resiliente face às técnicas mais avançadas de censura na Internet, protegendo, assim, o direito fundamental à liberdade de expressão num mundo digital cada vez mais sujeito a monitorização.

\keywords{
    Redes de Anonimato
    \and Privacidade Diferencial
    \and Comunicação Preservadora de Privacidade
    \and Ataques de Análise de Tráfego
    \and Garantias Reforçadas de Privacidade Ponto a Ponto 
    \and Correlação de Tráfego
}


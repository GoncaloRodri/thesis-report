%!TEX root = ../template.tex
%%%%%%%%%%%%%%%%%%%%%%%%%%%%%%%%%%%%%%%%%%%%%%%%%%%%%%%%%%%%%%%%%%%%
%% abstract-en.tex
%% NOVA thesis document file
%%
%% Abstract in English([^%]*)
%%%%%%%%%%%%%%%%%%%%%%%%%%%%%%%%%%%%%%%%%%%%%%%%%%%%%%%%%%%%%%%%%%%%

\typeout{NT FILE abstract-en.tex}%

The internet has become an indispensable tool for accessing information and exercising freedom of speech, a fundamental right that empowers individuals worldwide. However, authoritarian regimes and government agencies tend to suppress this right through surveillance and censorship activities. These efforts have become increasingly sophisticated, employing techniques such as traffic analysis attacks and advanced traffic correlation methods to monitor and undermine user anonymity.

In response to the growing demand for secure and private communication, Anonymity Networks like Tor have emerged as essential tools for safeguarding online end-to-end communication and fighting censorship, which millions of people rely on daily. However, recent research demonstrated that potential de-anonymization attacks can be performed by powerful state-level adversaries by traffic correlation and fingerprinting using advanced deep machine learning techniques.

This dissertation addresses these challenges by strengthening Tor network's defenses against the above-mentioned attacks. The main goal is to develop an innovative solution targeted to improve Tor relay nodes leveraging the existing software architecture by introducing formal privacy guarantees based on differentially private circuits. We expect that our proposed solution will maintain practical levels of throughput and latency for end-to-end communication, ensuring compatibility with real-world usage scenarios while providing users with adaptable and more robust anonymity protection.

As stated above, the core of the proposed solution is the integration of Differential Privacy principles to dynamically introduce carefully bounded random noise into traffic flows, mitigating correlation attacks and empowering a better solution resilient against fingerprinting. 
To the best of our knowledge, the expected dissertation contributions can represent a pioneering effort in incorporating Differential
Privacy into the software architecture of Tor relay nodes. Moreover, the use of Differential Privacy will allow for a formally proven, scalable, and effective mechanism that bolsters online anonymity, resistant against state-of-the-art internet censorship techniques and upholding the fundamental right to free expression in an increasingly monitored digital world.

\keywords{
    Anonymity Networks 
    \and Differential Privacy 
    \and End-to-End Privacy Enhanced Guarantees 
    \and Privacy-Preserving Communication 
    \and Traffic Analysis Attacks
    \and Traffic Correlation
}

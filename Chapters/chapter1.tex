%!TEX root = ../template.tex
%%%%%%%%%%%%%%%%%%%%%%%%%%%%%%%%%%%%%%%%%%%%%%%%%%%%%%%%%%%%%%%%%%%
%% chapter1.tex
%% NOVA thesis document file
%%
%% Chapter with introduction
%%%%%%%%%%%%%%%%%%%%%%%%%%%%%%%%%%%%%%%%%%%%%%%%%%%%%%%%%%%%%%%%%%%

\typeout{NT FILE chapter1.tex}%

\chapter{Introduction}\label{cha:introduction}

Freedom of speech and access to information are fundamental rights that many people take for granted. However, in some regions of the world, these rights are denied due to powerful and oppressive regimes and state level adversaries with access to vast resources that control the flow of information and restrict the access to the internet, in order to control the population~\cite*{aryan2013internet, zittrain2017shifting, alimardani2018internet}. 

To protect these populations and ensure their access to freedom of speech and information, `anonymity networks' are systems designed to preserve user anonymity while protecting their privacy and security. 
Authoritarian regimes intensify their efforts to restrict and control their population access by employing advanced techniques. 
Consequently, these networks are becoming increasingly important to protect the privacy and security of users, as well as the privacy and anonymity guarantees offered by these networks.

These regimes resort to the use of traffic analysis techniques to passively monitor the traffic and infer information about the users, in order to enforce and control the network and their users. 
Traffic analysis attacks are a significant threat to anonymity networks, as they can be used to infer users' activities and deny their privacy, as well as de-anonymize them~\cite*{chakravarty2014trafficanalysis, winter2012great, robjansen2019dosontor}.
Some have demonstrated that advances in machine learning lead to more sophisticated and effective traffic correlation attacks~\cite*{DeepCorr, TorMarker, TirDeanon}, such as DeepCoFFEA~\cite{DeepCoFFEA} and FINN~\cite{FINN}, which leverage deep learning techniques to correlate traffic patterns and infer users' activities with high accuracy, in anonymity networks like Tor~\cite{johnson2013users}.
Fingerprinting attacks also represent a major threat, as they enable attackers to infer information about users' online activities—even in encrypted traffic—by eavesdropping and collecting side-channel information between the user and the entry node.
Using this information, an attacker can apply machine learning techniques to infer a user's online activities—such as visited websites—by analyzing traffic patterns with high accuracy~\cite*{DeepFingerprinting, TikTok, OnlineWebFingerprinting}.

As attacks grow more sophisticated, journalists, whistleblowers, and anyone striving to exercise their freedom of speech in oppressive regions face increasing risks of identification and persecution. This not only diminishes the utility of anonymity networks but also fuels mistrust and fear. To combat these threats, the scientific community must develop robust solutions that strengthen anonymity guarantees, backed by proven privacy and security measures.


\section{Motivation}\label{sec:motivation}

Tor~\cite{dingledine2004tor}, one of the most popular anonymity networks, is an anonymous communication service based on the Onion Routing Protocol. This network is designed to protect users' privacy and security by routing their traffic through a network of volunteer-operated servers, encrypting it at each step, and ensuring that no single server knows both the source and destination of the traffic.

As mentioned earlier, anonymity networks remain vulnerable to attacks such as traffic analysis, which threatens the anonymity and security of the users these networks are designed to protect~\cite*{chakravarty2014trafficanalysis, winter2012great, robjansen2019dosontor, StatDisclosure, PracticalStatDisclosure, DeepFingerprinting, TikTok, OnlineWebFingerprinting, TrafficAnalysisLowMixnet, AnalysisMixNetsYeZhu}. To address the challenge of circumventing censorship, researchers have proposed a range of solutions. 

Some of these proposed sophisticated methods include Traffic Encapsulation, Pluggable Transports, \(k\)-Anonymity, Multipath Strategies, and Traffic Splitting. Traffic encapsulation is a method that conceals data by embedding packets within other data formats or protocols, thereby obfuscating their true content. This technique leverages protocols, such as those for web browsing and video streaming, to disguise network traffic, making it indistinguishable~\cite*{TorKameleon, MIRACE, StegoTorus, Protozoa, Stegozoa, Freewave}. Pluggable Transports serve as a mechanism for obfuscating network traffic by employing diverse protocols to connect to the Tor network, thereby complicating an attacker's ability to identify and analyze the traffic~\cite*{PlugTrans,Circumvention,TorKameleon}.@\(k\)-Anonymity is a formal privacy definition designed to ensure that an individual's data remains indistinguishable within a larger set~\cite*{DP_Book,KAnonSweeney,kAnonymityEffectiveness}. Meanwhile, Multipath Strategies and Traffic Splitting enhance security by distributing data across multiple concurrent paths, reducing the risk of interception and traffic analysis~\cite*{TorKameleon,MIRACE,Loopix,Wang_2022}. Similarly, traffic shuffling is a technique that focus on reordering packets, with statistical or cryptographic techniques, to make it difficult for an attacker to correlate the traffic~\cite*{DAENet, Karaoke, Riposte, Loopix, Dissent}. 

Some of these solutions have presented promising results in terms of privacy and security. However, we argue that the privacy guarantees provided by these solutions are not quantifiable, potentially making it difficult to evaluate their effectiveness and to compare them with other solutions. To address this problem, we apply Differential Privacy (DP)~\cite*{DifPrivacy,DifPrivacyCalNoise,DP_Book} to the Tor network in order to enhance its resistance against fingerprinting and correlation attacks, thereby provide quantifiable formal privacy guarantees to users, while maintaining practical levels of performance. DP was originally designed for database analysis and privacy preserving analysis but some works on communications systems have shown that DP can also be used to enhance privacy~\cite*{Loopix, Stadium, VilalongaINForum, StatPrivStreaming, NetShaper}. 


\section{Problem}\label{sec:problem}
Although anonymity networks are designed to provide robust privacy and security, advancements in traffic analysis and correlation techniques have increasingly exposed vulnerabilities, reducing their reliability and trustworthiness for users — particularly in regions where pervasive surveillance by authoritarian regimes is a critical concern. These regimes actively invest in sophisticated surveillance methods and traffic manipulation strategies to maintain control over digital communication~\cite*{aryan2013internet, zittrain2017shifting, alimardani2018internet}.
Tor~\cite{dingledine2004tor}, despite being one of the most widely used anonymity networks, is not immune to these challenges. It remains vulnerable to several attacks, including traffic correlation, congestion-based attacks, timing-based correlations, and fingerprinting techniques~\cite*{chakravarty2014trafficanalysis, johnson2013users,winter2012great, robjansen2019dosontor}. Such vulnerabilities highlight the ongoing arms race between those striving to safeguard online anonymity and those seeking to undermine it through technical and infrastructural advancements, such as deep learning techniques.

The absence of standardized metrics for evaluating privacy guarantees in a formal and rigorous manner raises concerns about the effectiveness of anonymizing networks like Tor. Research has demonstrated vulnerabilities in the Tor network to various attacks, including traffic analysis. For instance,~\citeauthor{chakravarty2014trafficanalysis} showed that statistical correlation techniques can be used to perform successful traffic analysis attacks~\cite{chakravarty2014trafficanalysis}. Additionally,~\citeauthor{yi2009fingerprinting} demonstrated that user privacy can be compromised through fingerprinting attacks~\cite{yi2009fingerprinting}.

\section{Goals}\label{sec:goals}

As stated previously in Section~\ref{sec:motivation}, it is important to ensure that anonymity networks are secure against threats like traffic fingerprinting and correlation attacks, empowered by deep learning techniques, and to address the lack of standardized privacy metrics to quantify the anonymity guarantees provided by these networks.

To address this challenge, we incorporate Differential Privacy (DP) into the Tor open-source project, by integrating Differential Privacy-based mechanisms in the existing software architecture of Tor relay node, in order to dynamically inject carefully bounded random noise into traffic flows to provide formal privacy guarantees to users. 
Our work builds upon the foundation laid by~\citeauthor{KIST}~\cite{KIST}, who introduced KIST, a congestion control scheduler for Tor designed to enhance network performance by reducing latency and increasing throughput. While KIST achieves significant efficiency gains, its focus remains solely on performance, without introducing additional privacy guarantees or defenses against traffic analysis attacks beyond those inherent to Tor. Inspired by this limitation, we extend KIST's approach, not only preserving its performance benefits but also reinforcing privacy protections and bolstering resistance to traffic analysis attacks.

In this manner, we reinforce the Tor end-to-end communication anonymity guarantees, even under traffic fingerprinting and correlation attacks. To ensure real-world applicability of this solution, it is important to maintain practical levels of throughput and latency for end-to-end communication. 

\subsection{Expected Contributions}\label{sub:contributions}

To address the stated goal, we produce the following contributions\footnote{The dissertation effort is planned as a research project task within NOVA LINCS (Nova Laboratory of Informatics and Computer Science), Computer Systems Research Group. The elaboration phase will involve integrating the expected contributions with collaborative engagement and follow-up of two PhD students: João Afonso Vilalonga and Hugo Gamaliel Pereira.}:

\begin{enumerate}
    \item Definition and formalization of the system design model to incorporate differentially private-based circuits as a possible interesting solution for Tor relay nodes to improve Tor anonymity conditions against traffic fingerprinting and correlation attacks;

    \item Specification of the software architecture and modular components for the engineering effort using Differential Privacy-based mechanisms and related components in extended Tor nodes;
   
    \item Prototyping of the proposed solution where the implemented nodes will be used in a validation test bench and publishing the prototype in open-source for the possible use of interested researchers and practitioners;
   
    \item Validation of the designed solution and prototype considering three main pillars: (\emph{i}) Development of formal security proofs behind the designed solution; (\emph{ii}) conduct an extensive experimental on performance indicators including communication latency, throughput, as well as resources' usage for running the proposed nodes, and (\emph{iii}) experimental observation of the unobservability properties of the designed solution against traffic correlation and fingerprinting techniques. 
   
\end{enumerate}

In the final validation effort, we obtained comparative metrics using a test bench that compared our designed solution with the existing Tor network and current Tor relay nodes.


\section{Report Organization}\label{sec:report_organization}
The remainder of this report is organized as follows: Chapter~\ref{cha:related_work} introduces anonymity networks, Tor and Differential Privacy, as well as some other related topics. The proposal of work is presented on Chapter~\ref{cha:system_model}, which provides some system model and guidelines for the development and discusses the proof method to verify the work. Then, we explore the implementation details, techniques and technologies used to develop the proposed solution in Chapter~\ref{cha:implementation}. Chapter~\ref{cha:evaluation} presents the evaluation of the proposed solution, including the experimental setup, performance evaluation, and security analysis. Finally, we conclude the report in Chapter~\ref{cha:conclusion}, summarizing the main findings and contributions of this work, and suggesting directions for future research.